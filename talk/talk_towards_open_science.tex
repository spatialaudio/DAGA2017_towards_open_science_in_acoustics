\documentclass{beamer}
%\documentclass{intbeamer}

% ===== packages =====
\usepackage{etex}
\usepackage[utf8]{inputenc}
\usepackage{graphicx}
\usepackage{tikz}
\usetikzlibrary{positioning, shadows, shapes}
\usepackage{amssymb}



% ===== beamer options =====
\setbeamertemplate{navigation symbols}{}
\graphicspath{{./figs/}}

% ===== titlepage info =====
\title[Towards Open Science in Acoustics]{\huge Towards Open Science \\ in Acoustics}
\subtitle{Foundations and Best Practices}

\author[Spors et al.]{Sascha Spors~$^1$, Matthias Geier~$^1$ and Hagen Wierstorf~$^2$}

\institute[]{$^1$ Institute of Communications Engineering, University of Rostock \\
$^2$ Filmuniversität Babelsberg \emph{KONRAD WOLF}}

\date[7.3.2017]{Jahrestagung der Deutschen Gesellschaft für Akustik \\ 7.3.2017 \\[4ex] \includegraphics[scale=.5]{CC_BY4png.png}}

% ===== macros =====
\tikzstyle{box}=[draw=structure.fg!60, fill=structure.fg!50, thick, drop shadow, rounded corners=4pt, minimum width=35mm, minimum height=17mm, anchor=east]
\tikzstyle{box2}=[draw=structure.fg!60, fill=structure.fg!50, thick, drop shadow, rounded corners=4pt, minimum width=35mm, minimum height=12mm, anchor=east]

\newcommand\data[1]{{\color{structure.fg}#1}}

\newcommand\OSstamp[1]{%
\begin{tikzpicture}
\node[box2] (methodology) {#1};
\end{tikzpicture}}

\newcommand\OM{\OSstamp{Open Methodology}}
\newcommand\OS{\OSstamp{Open Source}}
\newcommand\OD{\OSstamp{Open Data}}
\newcommand\OA{\OSstamp{Open Access}}
\newcommand\OPR{\OSstamp{Open Peer Review}}




% =============================================================================
% =============================================================================
\begin{document}

\maketitle

% =============================================================================
\begin{frame}{Who Should Benefit from my Research?}

\begin{columns}[T]
\begin{column}{.55\linewidth}

\begin{itemize}
\item[$\square$] myself
\item[$\square$] my future self
\item[$\square$] my boss
\item[$\square$] my colleagues
\item[$\square$] other researchers
\item[$\square$] all people in the world
\item[$\square$] science itself
\end{itemize}

\end{column}
%
\begin{column}{.35\linewidth}

\begin{tikzpicture}
\draw[shading = axis, top color=structure.fg!30!white, bottom color=structure.fg] (-.75,-5) -- (.75,-5) -- (0,0) -- cycle;
\node at (1.75,-2.5) {\shortstack{Ease of \\ Reproducibility}};
\end{tikzpicture}

\end{column}
\end{columns}

\end{frame}


% =============================================================================
\begin{frame}{Reproducibility of a Listening Experiment}
\framesubtitle{1. Idea}


\includegraphics[scale=.2]{./figs/Bright-Idea-800px}

{\tiny from https://openclipart.org/}

\end{frame}

% =============================================================================
\begin{frame}[noframenumbering]{Reproducibility of a Listening Experiment}
\framesubtitle{2. Design of Experiment}

\begin{columns}
\begin{column}{.45\linewidth}
\begin{itemize}
\item hypothesis
\item design of listening experiment
\end{itemize}
\end{column}
%
\begin{column}{.45\linewidth}
\OM
\end{column}
\end{columns}

\end{frame}


% =============================================================================
\begin{frame}[noframenumbering]{Reproducibility of a Listening Experiment}
\framesubtitle{3. Computation and Implementation}

\begin{columns}
\begin{column}{.45\linewidth}
\begin{itemize}
\item mathematical derivations
\item implementation of simulations, signal processing
\item implementation of control logic, graphical user interface
\end{itemize}

\begin{itemize}
\item numerical simulations
\item generation of stimuli
\end{itemize}
\end{column}
%
\begin{column}{.45\linewidth}
\OD
\vspace{5mm}
\OS
\end{column}
\end{columns}
\end{frame}


% =============================================================================
\begin{frame}[noframenumbering]{Reproducibility of a Listening Experiment}
\framesubtitle{4. Experiment}

\begin{columns}
\begin{column}{.45\linewidth}
Illustration of experiment...

Output: raw data
\end{column}
%
\begin{column}{.45\linewidth}
\OD
\end{column}
\end{columns}



\end{frame}


% =============================================================================
\begin{frame}[noframenumbering]{Reproducibility of a Listening Experiment}
\framesubtitle{5. Analysis}

\begin{columns}
\begin{column}{.45\linewidth}
\begin{itemize}
\item anonymization of data
\item outlier removal
\item statistical analysis
\end{itemize}

Output: processed data
\end{column}
%
\begin{column}{.45\linewidth}
\OM
\vspace{5mm}
\OD
\vspace{5mm}
\OS
\end{column}
\end{columns}

\end{frame}


% =============================================================================
\begin{frame}[noframenumbering]{Reproducibility of a Listening Experiment}
\framesubtitle{6. Manuscript}

\begin{columns}
\begin{column}{.45\linewidth}
\begin{itemize}
\item text
\item references
\item visualization of results (plots)
\item supplementary material
\end{itemize}
\end{column}
%
\begin{column}{.45\linewidth}
\OS
\vspace{5mm}
\OA
\end{column}
\end{columns}

\end{frame}

% =============================================================================
\begin{frame}[noframenumbering]{Reproducibility of a Listening Experiment}
\framesubtitle{7. Pre-Publication Peer Review}

\begin{columns}
\begin{column}{.45\linewidth}
\begin{itemize}
\item ratings, comments
\item revision of manuscript
\end{itemize}
\end{column}
%
\begin{column}{.45\linewidth}
\OPR
\end{column}
\end{columns}

\end{frame}

% =============================================================================
\begin{frame}[noframenumbering]{Reproducibility of a Listening Experiment}
\framesubtitle{8. Publication}

\begin{columns}
\begin{column}{.45\linewidth}
\includegraphics[page=1,width=120pt]{/Users/spors/Documents/publications/published/DAGA_2015/paper/DAGA2015_WFS_edge_effects.pdf}
\end{column}
%
\begin{column}{.45\linewidth}
\OA
\end{column}
\end{columns}

\end{frame}

% =============================================================================
\begin{frame}[noframenumbering]{Reproducibility of a Listening Experiment}
\framesubtitle{9. Aftermath}

\begin{columns}
\begin{column}{.45\linewidth}
\begin{itemize}
\item reproduction by third parties
\item post-publication review
\item errata, code revision
\end{itemize}
\end{column}
%
\begin{column}{.45\linewidth}
\OA
\end{column}
\end{columns}

\end{frame}


% =============================================================================
\begin{frame}{The Scientific Method}

\textbf{Branches of the Scientific Method} {\tiny [Donoho 2009]}
\begin{enumerate}
\item deductive $\rightarrow$ mathematics, formal logic
%
\item empirical $\rightarrow$ statistical analysis of controlled experiments
%
\item computational
\begin{itemize}
\item large-scale simulations
\item data-driven computational science
\end{itemize}
\end{enumerate}

\vfill

\textbf{Types of Reproducibility} {\tiny [Stodden 2014a]}
\begin{itemize}
\item empirical reproducibility
\item computational reproducibility
\item statistical reproducibility
\end{itemize}

\end{frame}


% =============================================================================
\begin{frame}{Open Science}


\begin{tikzpicture}[node distance=5mm and 2mm]
\node[box] (methodology) {Open Methodology};
\node[box, right=of methodology] (data) {Open Data};
\node[box, right=of data] (source) {Open Source};
\node[box, below=of methodology] (access) {Open Access};
\node[box, below=of data] (edu) {\shortstack{Open Educational \\ Resources}};
\node[box, below=of source] (review) {Open Peer Review};
\end{tikzpicture}

\vspace{2mm}
{\tiny adapted from [\url{https://en.wikipedia.org/wiki/Open_science}, \url{http://openscienceasap.org/open-science/]}}

\end{frame}


% =============================================================================
\begin{frame}{Reproducibility of a Listening Experiment Revisited}


\end{frame}


% =============================================================================
\begin{frame}{Incentives and Barriers}
\framesubtitle{Selected Results from a Survey of the Machine Learning Community}

\textbf{Barriers}(\data{Data}/Code) {\tiny [Stodden 2010], N=134}
\begin{itemize}
\item time to document and clean up (\data{54}/77 \%)
\item dealing with questions from users (\data{34}/52 \%)
\item not receiving attribution (\data{44}/42 \%)
\item possibility of patents (\data{--}/40 \%)
\item legal barriers (e.g. copyright) (\data{34}/41 \%)
\end{itemize}

\vspace{3mm}

\textbf{Incentives}(\data{Data}/Code)
\begin{itemize}
\item encourage scientific advancement (\data{81}/91 \%)
\item encourage sharing in others (\data{90}/79 \%)
\item be a good community member (\data{86}/79 \%)
\item set a standard in the field (\data{82}/76 \%)
\item improve the calibre of research (\data{85}/74 \%)
\end{itemize}

\end{frame}


% =============================================================================
\begin{frame}{Management of Research Data}

\begin{itemize}
\item systematic management of research data is a prerequisite for open \\ and reproducible science
\item becoming mandatory in funding schemes (DFG, Horizon 2020, NSF, ...)
\end{itemize}

\vfill

\textbf{Principles} {\tiny [DFG 2013, HRK 2014, Stodden 2014b, H2020 2016]}
\begin{itemize}
\item develop a comprehensive data management plan
\item use workflow tracking in the research process
\item make data findable, accessible, interoperable and reusable (FAIR)
\item apply open licensing models
\item offer training and qualification
\end{itemize}

\end{frame}


% =============================================================================
\begin{frame}{Public Services for Open Science}

\begin{columns}[T]
\begin{column}{.45\linewidth}
\textbf{Generic Repositories}
\begin{itemize}
\item GitHub
\item Bitbucket
\end{itemize}
\end{column}
%
\begin{column}{.45\linewidth}
\textbf{Repositories for Research Data}
\begin{itemize}
\item Zenodo
\item runmycode
\item datahub
\end{itemize}
\end{column}
\end{columns}

\vfill

\textbf{Virtual Research Environments}
\begin{itemize}
\item Open Science Framework (OSF)
\item gCUBE
\item hubzero
\end{itemize}

\vfill

\textbf{Journals}
\begin{itemize}
\item Open Science Journal
\item Journal of Open Research Software
\end{itemize}

\end{frame}


% =============================================================================
\begin{frame}{Copyright and Licenses}


\begin{itemize}
\item unclear situation when publishing data without explicit license
\item license should be as open as possible in order to promote re-use
\item legal implications are complex and hard to oversee
\end{itemize}

\vfill

\textbf{Available Licensing Frameworks}
\begin{itemize}
\item Software: GNU Public License, BSD, MIT, ...
\item Content: Creative Commons, ...
\end{itemize}

\textbf{Recommendations}
\begin{itemize}
\item Reproducible Research Standard (RRS) [Stodden, 2009]
\end{itemize}

\end{frame}


% =============================================================================
\begin{frame}{Personal Experience}

\begin{itemize}
\item public release of the SoundScape Renderer (SSR) in 2010
\item various toolboxes, datasets, open educational resources
\item internal data management: Redmine, svn, git
\item public releases: github, zenodo, wordpress
\end{itemize}

\vfill

\textbf{Benefits}
\begin{itemize}
\item documentation/clean up/discussions for public release
\item bug reports
\item positive community feedback
\item potentially more citations {\tiny [?]}
\end{itemize}

\textbf{Challenges}
\begin{itemize}
\item initial effort (training, ...)
\item missing versioning tool/platform for (large) data bases
\end{itemize}

\end{frame}


% =============================================================================
\begin{frame}{Conclusions}

\begin{itemize}
\item reproducibility of results is essential for the scientific method
\item Open Science by itself does not ensure the ease of reproducibility
\item evaluation measures contradict scientific innovation
\item training and qualification is essential
\end{itemize}

\vfill

{\large
\url{https://github.com/spatialaudio}\\[1ex]
\url{https://github.com/twoears}\\[1ex]
\url{http://spatialaudio.net}
}

\end{frame}


% =============================================================================
\begin{frame}{References}

{\scriptsize
\begin{description}
\item[{[Donoho 2009]}] David Donoho, Arian Maleki, Inam Rahman, Morteza Shahram, Victoria Stodden, 15 Years of Reproducible Research in Computational Harmonic Analysis, Computing in Science and Engineering, 11(1), 2009.
\item[{[DFG 2013]}] Sicherung guter wissenschaftlicher Praxis, Deutsche Forschungsgemeinschaft, 2013. 
\item[{[HRK 2014]}] Hochschulrektorenkonferenz. Management von Forschungsdaten - eine zentrale strategische Herausforderung für Hochschulleitungen, 13.5.2014.
\item[{[H2020 2016]}] H2020 Programme: Guidelines of FAIR Data Management in Horizon 2020, European Research Commission, 26.7.2016.
\item[{[Stodden 2009]}] Vitoria Stodden, The Legal Framework for Reproducible Scientific Research, Computing in Science \& Engineering, January/February 2009.
\item[{[Stodden 2010]}]
\item[{[Stodden 2014a]}] Victoria Stodden, Resolving Reproducibility in Computational Science: Tools, Policy, and Culture, talk, 8.10.2014.
\item[{[Stodden 2014b]}] Victoria Stodden and Sheila Miguez, Best Practices for Computational Science: Software Infrastructure and Environments for Reproducible and Extensible Research, Journal of Open Reserach Software, 2(1):e21, pp. 1-6.


\end{description}
}

\end{frame}


\end{document}