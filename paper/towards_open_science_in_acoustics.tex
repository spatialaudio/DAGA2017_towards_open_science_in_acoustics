\documentclass[a4paper, 10pt, twocolumn]{article}

\usepackage[utf8]{inputenc}         
\usepackage{graphicx}
\usepackage{url}
\usepackage[small,bf]{caption2}
\usepackage{parskip}
\usepackage{titlesec}
\usepackage{amsmath, amssymb}

\titleformat{\section}{\normalfont\large\bfseries}{\thesection}{}{}
\titleformat{\subsection}{\normalfont\large\bfseries}{\thesection}{}{}
\titleformat{\paragraph}{\normalfont\bfseries}{\theparagraph}{}{}
\titlespacing{\section}{0pt}{6pt}{-1pt}
\titlespacing{\subsection}{0pt}{3pt}{-1pt}
\titlespacing{\paragraph}{0pt}{3pt}{-1pt}

% Definition der Seitenränder
\addtolength{\textwidth}{2.1cm}
\addtolength{\topmargin}{-2.4cm}
\addtolength{\oddsidemargin}{-1.1 cm}
\addtolength{\textheight}{4.5cm}
\setlength{\columnsep}{0.7cm}

\pagestyle{empty}                   % weder Kopf- noch Fußzeile auf 1. Seite


\date{}                                         % kein Datum auf 1. Seite

\title{\vspace{-8mm}\textbf{\large
Towards Open Science in Acoustics: Foundations and Best Practices}}

% Hier die Namen und Daten der beteiligten Autoren eintragen
\author{
Sascha Spors$^1$, Matthias Geier$^1$ and Hagen Wierstorf$^2$\\
$^1$ \emph{\small Institute of Communications Engineering, University of Rostock, Germany, Email: sascha.spors@uni-rostock.de}\\
$^2$ \emph{\small Filmuniversität Babelsberg KONRAD WOLF}} 


% ================================================================================
% ================================================================================
\begin{document}

\maketitle
\thispagestyle{empty}           % weder Kopf- noch Fußzeile auf Folgeseiten

% ================================================================================
\section*{The Scientific Method} \label{sec:intro} 

Before discussing the Open Science approach in detail its worthwhile to review the
foundations of the scientific method underlying research results. It may be defined 
as \emph{'A method of procedure that has characterized natural science since the 17th 
century, consisting in systematic observation, measurement, and experiment, and the 
formulation, testing, and modification of hypotheses.'}~\cite{scientific_method:OXD}.
The reproducibility of results is one of the main principles of the scientific method. 
It may be defined as \emph{'Reproducibility is the ability of an entire analysis of 
an experiment or study to be duplicated, either by the same researcher or by someone 
else working independently, whereas reproducing an experiment is called replicating 
it.'}~\cite{Leek15:PNAS}. 
The irreproducibility of a wide range of scientific results has drawn significant 
attention in the last decade~\cite{Borgwardt:Book,retraction:WWW,ioannidis05:PLOS, open15:AAAS, chalmers09:OG, freedman15:PLOS, howells14:Nature}. 
In order to track down the problem the application areas of the scientific method have been
classified into the three following branches~\cite{Donoho:CSE, Stodden2014:talk}:
\begin{enumerate}
\item deductive
\item empirical
\item computational
\end{enumerate}
The deductive branch covers results derived e.g. by formal logic and mathematics, the empirical
branch e.g. statistical analysis of controlled experiments. The first two are traditional branches, 
while the last one is a potentially new branch. It covers results derived by large-scale simulations
and data-driven computational science. The measures that have to be taken to ensure reproducibilitly
in the traditional are quite well known. This does not hold for the third, computational, branch.

Besides problems in the research methods themselves, results are often not reproducible since 
necessary supplementary material as protocols, data and implementations are not available. In 
many cases only the published results are available to other researchers. Open Science 
focuses on the ease of access of scientific data and therefore supports the ease of 
reproducibility.

An alternative view on the problem of reproducibility is given by asking 
\textbf{who should benefit from my research}? Potential answers to this question ordered by 
the degree of required reproducibilitly are:
\begin{itemize}
\item[$\square$] myself
\item[$\square$] my future self
\item[$\square$] my colleagues
\item[$\square$] other researchers
\item[$\square$] all people in the world
\item[$\square$] science itself
\end{itemize}


% ================================================================================
\section*{Open Science} \label{sec:open_science} 

Open Science bases on the general demand for socialization of knowledge. More specifically, 
it aims at making the research process transparent on all levels. The advent of scientific 
journals can be seen as a first step towards openness. Open Science has been discussed on
a broad level in the last decade. However the exact meaning of the term is still not fully 
settled. Many measures are discussed to achieve transparency in the research process~\cite{Pontika15:ACM,Reproducibility15:AMS,Albagli15:Book,Vrana15:MIPRO,Kraker11:TEL}. 
We review the most common elements in the following, followed by application examples in
the context of a listening experiment.


% ................................................................................
\paragraph*{Elements of Open Science}

The various measures used to gain open and transparent science may be classified as
follows:
\begin{itemize}
\item \textbf{Open Source}\\
Availability of source code of implementations used in the research process, e.g. for numerical simulations.
%
\item \textbf{Open (Science) Data}\\
Availability of the data underlying the research, whereas data does not only refer to electronic resources but also
to protocols, samples, etc.
%
\item \textbf{Open Access}\\
Free access to published research output, e.g. articles.
%
\item \textbf{Open Methodology}\\
Detailed documentation of the methodology underlying the research.
%
\item \textbf{Open Notebook Science}\\
Availability of the primary record of research results, e.g. laboratory notebooks or detailed mathematical derivations.
%
\item \textbf{Open Educational Resources}\\
Free accessible resources for teaching, learning and research.
\item \textbf{Open Peer Review}~\cite{Ford13:LFP}\\
Refers to various transparency in the peer review process, e.g. crowdsourced review.
\end{itemize}


% ................................................................................
\paragraph*{Example -- Listening Experiment}

The workflow of a typical perceptual experiment serves as an example for the usage of 
the different elements of open science. A perceptual study may be decomposed into the
following stages:

\begin{enumerate}
\item \textbf{Idea}\\
The results of a preceding experiment or a discussion among colleagues is often the advent 
for new research projects. These ideas reveal for instance the motivation behind the research. 
Publishing the ideas as \underline{Open Notebook Science} makes them accessible to the public.
%
\item \textbf{Design of Experiment}\\
The experiment is designed on basis of the initial idea. This involves for instance the formulation 
of a hypothesis, the definition of a procedure, stimuli and subjects. \underline{Open Methodology} makes
the design available.
%
\item\textbf{Computation}\\
The generation of stimuli may involve mathematical deviations for numerical simulations, the 
implementation of signal processing, control logic and graphical user interface, as well as input data. These
can be made available in the sake of \underline{Open Notebook Science}, \underline{Open Data} and \underline{Open Source}.
%
\item \textbf{Experiment}\\
The experiment is then conducted in the next step and the responses are collected from the test subject. This
constitutes raw data which is processed later. The data can be published as \underline{Open Data} as long as
no privacy issues are raised.
%
\item \textbf{Analysis}\\
The statistical analysis of the raw data is the basis of the generalization of the individual results. This may
involve the anonymization and outlier removal as first steps. The analysis can be published using the elements
\underline{Open Methodology}, \underline{Open Source} and \underline{Open Data}.
%
\item \textbf{Manuscript}\\
The design and the results of the perceptual study are compiled into a manuscript for publication. It is composed
from text, references and visualization of the analyzed results. The manuscript may be published as \underline{Open Access}.
%
\item \textbf{Pre-Publication Peer Review}\\
The publication process typically involves some kind of quality assessment. The manuscript is evaluated by
independent researchers who provide ratings and suggestions for improvement. If accepted the manuscript is
the revised on this basis. The peer review process can be made transparent using the elements of \underline{Open Peer Review}.
%
\item \textbf{Publication}\\
After successful incorporation of comments from the pre-publication peer review the manuscript is ready for
publication. It may be accompanied by supplementary material in order to improve the ease of reproducibility. In the
case of conference papers a presentation may be given which contains other or additional. The elements involved
for publication are \underline{Open Access}, \underline{Open Source} and \underline{Open Data}.
%
\item \textbf{Aftermath}\\
After publication, the presented research may be replicated by others. Post-publication
review will rate the significance of the research. Feedback from this
may require to publish an errata or revise the underlying data and code. Finally open questions
may bring up ideas for the next research project.
\end{enumerate}

% ================================================================================
\section*{Copyright and Licenses} \label{sec:copyright}


% ================================================================================
\section*{Management of Research Data} \label{sec:data_management}

Principles \cite{DFG_GWP:Book,HRK_FDM:WWW,Stodden2014:JORS,H2020_FAIR:ERC}
\begin{itemize}
\item develop a comprehensive data management plan
\item use workflow tracking in the research process
\item make data findable, accessible, interoperable and reusable (FAIR)
\item apply open licensing models
\item offer training and qualification
\end{itemize}


% ================================================================================
\section*{Conclusions} \label{sec:conclusions} 

% ................................................................................
\paragraph*{Personal Experience}

% ................................................................................
\paragraph*{Studies}

\begin{itemize}
\item studies on benefits of open science/access
\end{itemize}

% ................................................................................
\paragraph*{Conclusions}

\begin{itemize}
\item reproducibility is essential for the scientific method
\item Open Science by itself does not ensure the ease of reproducibility
\item institutional evaluation measures contradict scientific innovation
\item training and qualification required
\end{itemize}


%================================================================================
% Bibliography
%================================================================================
{
\bibliographystyle{IEEEtran}
\bibliography{open_science}
}



\end{document}