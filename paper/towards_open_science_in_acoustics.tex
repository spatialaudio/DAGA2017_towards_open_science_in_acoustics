\documentclass[a4paper, 10pt, twocolumn]{article}

\usepackage[utf8]{inputenc}         
\usepackage{graphicx}
\usepackage{url}
\usepackage[small,bf]{caption2}
\usepackage{parskip}
\usepackage{titlesec}
\usepackage{amsmath, amssymb}

\titleformat{\section}{\normalfont\large\bfseries}{\thesection}{}{}
\titleformat{\subsection}{\normalfont\large\bfseries}{\thesection}{}{}
\titleformat{\paragraph}{\normalfont\bfseries}{\theparagraph}{}{}
\titlespacing{\section}{0pt}{6pt}{-1pt}
\titlespacing{\subsection}{0pt}{3pt}{-1pt}
\titlespacing{\paragraph}{0pt}{3pt}{-1pt}

% Definition der Seitenränder
\addtolength{\textwidth}{2.1cm}
\addtolength{\topmargin}{-2.4cm}
\addtolength{\oddsidemargin}{-1.1 cm}
\addtolength{\textheight}{4.5cm}
\setlength{\columnsep}{0.7cm}

\pagestyle{empty}                   % weder Kopf- noch Fußzeile auf 1. Seite


\date{}                                         % kein Datum auf 1. Seite

\title{\vspace{-8mm}\textbf{\large
Towards Open Science in Acoustics: Foundations and Best Practices}}

% Hier die Namen und Daten der beteiligten Autoren eintragen
\author{
Sascha Spors$^1$, Matthias Geier$^1$ and Hagen Wierstorf$^2$\\
$^1$ \emph{\small Institute of Communications Engineering, University of Rostock, Germany, Email: sascha.spors@uni-rostock.de}\\
$^2$ \emph{\small Filmuniversität Babelsberg KONRAD WOLF}} 


% ================================================================================
% ================================================================================
\begin{document}

\maketitle
\thispagestyle{empty}           % weder Kopf- noch Fußzeile auf Folgeseiten

% ================================================================================
\section*{Introduction} \label{sec:intro} 

The reproducibility of results is one of the main principles of the scientific method. 
The irreproducibility of a wide range of scientific results has recently drawn significant 
attention. Besides problems in the research methods themselves, results were often not 
reproducible since necessary supplementary material as protocols, data and implementations 
were not available. Another issue is the lacking availability of data for further 
research by third parties. In many cases only the published results are available to 
other researchers. Open Science focuses on the ease of access and reproducibility of 
scientific results. This contribution introduces the concept of reproducibility and 
addresses common concerns. Best practices for Open Science in acoustics research are 
discussed and illustrated at examples.


% ================================================================================
\section*{The Scientific Method} \label{sec:method}

\begin{itemize}
\item short intro into the scientific method
\item the three/four branches according to\cite{Donoho:CSE, Stodden2014:talk}
\item the problem of reproducibility
\item one potential countermeasure: open science
\end{itemize}

Alternative view: who should benefit from my research?

\begin{itemize}
\item[$\square$] myself
\item[$\square$] my future self
\item[$\square$] my colleagues
\item[$\square$] other researchers
\item[$\square$] all people in the world
\item[$\square$] science itself
\end{itemize}


% ================================================================================
\section*{Open Science} \label{sec:open_science} 

% ................................................................................
\paragraph*{Elements of Open Science}

The elements of open science \cite{}
\begin{enumerate}
\item Open Source
\item Open (Science) Data
\item Open Access
\item Open Methodology
\item Open Notebook Science
\item Open Educational Resources
\item Open Peer Review
\item Open Research (?)
\end{enumerate}

% ................................................................................
\paragraph*{Example}

Perceptual study as example of usage of open science elements:

\begin{enumerate}
\item Idea
\item Design of Experiment
\item Computation
\item Experiment
\item Analysis
\item Manuscript
\item Pre-Publication Peer Review
\item Publication
\item Aftermath
\end{enumerate}

% ================================================================================
\section*{Copyright and Licenses} \label{sec:copyright}


% ================================================================================
\section*{Management of Research Data} \label{sec:data_management}

Principles \cite{DFG_GWP:Book,HRK_FDM:WWW,Stodden2014:JORS,H2020_FAIR:ERC}
\begin{itemize}
\item develop a comprehensive data management plan
\item use workflow tracking in the research process
\item make data findable, accessible, interoperable and reusable (FAIR)
\item apply open licensing models
\item offer training and qualification
\end{itemize}


% ================================================================================
\section*{Conclusions} \label{sec:conclusions} 

% ................................................................................
\paragraph*{Personal Experience}

% ................................................................................
\paragraph*{Studies}

\begin{itemize}
\item studies on benefits of open science/access
\end{itemize}

% ................................................................................
\paragraph*{Conclusions}

\begin{itemize}
\item reproducibility is essential for the scientific method
\item Open Science by itself does not ensure the ease of reproducibility
\item institutional evaluation measures contradict scientific innovation
\item training and qualification required
\end{itemize}


%================================================================================
% Bibliography
%================================================================================
{
\bibliographystyle{IEEEtran}
\bibliography{open_science}
}



\end{document}